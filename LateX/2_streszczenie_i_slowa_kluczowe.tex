\begin{center}
    {\bf\Large{Streszczenie}}\\
\end{center}
Praca opisuje zaprojektowaną i wykonaną przez jej autorów implementację
aplikacji internetowej wspomaganej przez sztuczną inteligencję. Aplikacja dotyczy obszaru samodoskonalenia się i poprawiania swoich umiejętności w różnych dziedzinach życia. Aplikacja będzie stworzona w architekturze mikro serwisów oraz zbudowanie na podstawie najbardziej aktualnych wzorców projektowych.
Szczególny nacisk położono na integrację ze sztuczną inteligencją, która pomaga użytkownikowi aplikacji zaplanować plan rozwoju w zależności do jego preferencji.
Zespół składa się z programistów tworzących aplikację oraz inżyniera DevOps, automatyzującego procesy oraz planującego architekturę całego systemu. Jako programiści chcemy wykorzystać potencjał modeli językowych (LLM) w planowaniu codziennych zadań oraz obowiązków.
\\
{\bf Słowa kluczowe:} artificial intelligence, microservices, cloud, DevOps, LLM, fine tuning.
\clearpage

\begin{center}
    {\bf\Large{Abstrakt}}\\
\end{center}
The thesis describes the designed and implemented web application supported by artificial intelligence, created by its authors. The application focuses on the self-improvement and enhancement of skills in various life domains. It will be developed using a microservices architecture and built based on the most current design patterns. Special emphasis is placed on integration with artificial intelligence, which assists the application user in planning a development strategy according to their preferences.

The team consists of developers creating the application and a DevOps engineer automating processes and planning the architecture of the entire system. As developers, we aim to leverage the potential of language models (LLM) in planning daily tasks and responsibilities.
\\
{\bf Słowa kluczowe:} sztuczna inteligencja, mikroserwisy, chmura, DevOps, LLM, fine tuning.