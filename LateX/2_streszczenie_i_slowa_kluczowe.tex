\begin{center}
    {\bf\Large{Streszczenie}}\\
\end{center}
Praca opisuje implementację aplikacji internetowej wspomaganej przez sztuczną inteligencję. Aplikacja dotyczy obszaru samodoskonalenia się i poprawiania swoich umiejętności w różnych aspektach i dziedzinach życia. Aplikacja będzie stworzona w architekturze mikro serwisów oraz zbudowana na podstawie współczesnych wzorców projektowych.
Szczególny nacisk położono na integrację ze sztuczną inteligencją, której celem jest pomoc użytkownikowi aplikacji oraz utworzenie planu rozwoju w zależności indywidualnych preferencji.
\\
{\bf Słowa kluczowe:} sztuczna inteligencja, mikroserwisy, chmura, DevOps, LLM, fine tuning.
\clearpage

\begin{center}
    {\bf\Large{Abstrakt}}\\
\end{center}
The thesis describes the implementation of a web application supported by artificial intelligence. The application deals with the area of self-improvement and improving one's skills in various aspects and areas of life. The application will be developed in micro-services architecture and built on the basis of contemporary design patterns.
Special emphasis has been placed on integration with artificial intelligence, the purpose of which is to help the application user and create a development plan according to individual preferences.
\\
{\bf Keywords:} artificial intelligence, microservices, cloud, DevOps, LLM, fine tuning.


