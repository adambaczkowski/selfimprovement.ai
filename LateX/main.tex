\documentclass[12pt]{mwart}
\renewcommand{\baselinestretch}{1.5} 
\usepackage[utf8]{inputenc}
\usepackage{polski}
\usepackage{graphicx}
\usepackage{tocloft}

% ------------------------------------------------
%formatowanie kodu
\usepackage{listings}
\usepackage{xcolor}

% dodanie kolorów
\definecolor{codegreen}{rgb}{0,0.6,0}
\definecolor{codegray}{rgb}{0.5,0.5,0.5}
\definecolor{codepurple}{rgb}{0.58,0,0.82}
\definecolor{backcolour}{rgb}{0.95,0.95,0.92}

% definiowanie stylu kodu
\lstdefinestyle{mystyle}{
  backgroundcolor=\color{backcolour}, commentstyle=\color{codegreen},
  keywordstyle=\color{magenta},
  numberstyle=\tiny\color{codegray},
  stringstyle=\color{codepurple},
  basicstyle=\ttfamily\footnotesize,
  breakatwhitespace=false,         
  breaklines=true,                 
  captionpos=b,                    
  keepspaces=true,                 
  numbers=left,                    
  numbersep=5pt,                  
  showspaces=false,                
  showstringspaces=false,
  showtabs=false,                  
  tabsize=2
}

%"mystyle" code listing set
\lstset{style=mystyle}

% ------------------------------------------------

% Ustawienia akapitów
\setlength{\parindent}{1.5cm}

% Zmiana nazwy na "Spis treści"
\renewcommand*\contentsname{Spis treści}

\begin{document}

\begin{center}

    %University logo
    \includegraphics[width=1\textwidth]{Obrazy/PJATK_logo.png}
    \vspace{1cm}
    
     %Department
    {\normalsize Wydział Informatyki\par}
    \vspace{1cm}
    
    %Department
    {\normalsize Nazwa Katedry\par}
    {\bf{\normalsize Katedra Sieci Komputerowych i Systemów Rozproszonych\par}}
    \vspace{1cm}

    %Specialization
    {\normalsize Nazwa specjalizacji\par}
    {\bf{\normalsize Technologie sieci urządzeń mobilnych oraz chmury obliczeniowej\par}}
    \vspace{1cm}
    
    %Author's name
    {\bf\normalsize Grzegorz Święcicki,\\Adam Bączkowski,\\Mateusz Młodochowski\par}
    {\normalsize s20978, s20132, s29707\par}
    \vspace{0.8cm}
    
    %Thesis title
    {\bf\uppercase{\normalsize wykorzystanie sztucznej inteligencji w aplikacji internetowej do samodoskonalenia \par}}
    \vspace{1cm}
    
    %Thesis type
    {\raggedleft{\normalsize Praca magisterska \\ napisana pod opieką:\par}}
    {\raggedleft{\normalsize dr Pavel Savov\par}}
    {\raggedleft{\normalsize mgr. inż. Mokkas Mokkas\par}}
    \vspace{1cm}
    
    %Date
    {\normalsize Warszawa Lipiec 2024}

\end{center}
\clearpage

\begin{center}
    {\bf\Large{Streszczenie}}\\
\end{center}
Praca opisuje zaprojektowaną i wykonaną przez jej autorów implementację
aplikacji internetowej wspomaganej przez sztuczną inteligencję. Aplikacja dotyczy obszaru samodoskonalenia się i poprawiania swoich umiejętności w różnych dziedzinach życia. Aplikacja będzie stworzona w architekturze mikro serwisów oraz zbudowanie na podstawie najbardziej aktualnych wzorców projektowych.
Szczególny nacisk położono na integrację ze sztuczną inteligencją, która pomaga użytkownikowi aplikacji zaplanować plan rozwoju w zależności do jego preferencji.
Zespół składa się z programistów tworzących aplikację oraz inżyniera DevOps, automatyzującego procesy oraz planującego architekturę całego systemu. Jako programiści chcemy wykorzystać potencjał modeli językowych (LLM) w planowaniu codziennych zadań oraz obowiązków.
\\
{\bf Słowa kluczowe:} artificial intelligence, microservices, cloud, DevOps, LLM, fine tuning.
\clearpage

\begin{center}
    {\bf\Large{Abstrakt}}\\
\end{center}
The thesis describes the designed and implemented web application supported by artificial intelligence, created by its authors. The application focuses on the self-improvement and enhancement of skills in various life domains. It will be developed using a microservices architecture and built based on the most current design patterns. Special emphasis is placed on integration with artificial intelligence, which assists the application user in planning a development strategy according to their preferences.

The team consists of developers creating the application and a DevOps engineer automating processes and planning the architecture of the entire system. As developers, we aim to leverage the potential of language models (LLM) in planning daily tasks and responsibilities.
\\
{\bf Słowa kluczowe:} sztuczna inteligencja, mikroserwisy, chmura, DevOps, LLM, dostrajanie.

\tableofcontents

\section{Przedmowa}

\indent Przedstawiamy niniejszą pracę, która dokumentuje projektowanie i realizację aplikacji internetowej z wykorzystaniem wsparcia sztucznej inteligencji. Głównym celem tego projektu było opracowanie narzędzia umożliwiającego użytkownikom rozwój i doskonalenie w różnych dziedzinach życia. Zdecydowano się na wykorzystanie architektury mikroserwisów oraz najnowszych wzorców projektowych w celu zapewnienia efektywności, skalowalności i łatwości zarządzania systemem.

Szczególny nacisk położono na integrację sztucznej inteligencji, która pełni kluczową rolę w personalizacji planów rozwoju użytkowników. Aplikacja opiera się na zaawansowanych algorytmach, analizujących cele i preferencje użytkowników w celu proponowania spersonalizowanych strategii rozwoju.

Celem było nie tylko stworzenie funkcjonalnej aplikacji, ale również zapewnienie jej niezawodności, bezpieczeństwa i łatwości obsługi.

W trakcie realizacji projektu zdecydowano się na wykorzystanie potencjału modeli językowych (LLM) w celu usprawnienia planowania codziennych zadań i obowiązków. Dzięki temu użytkownicy będą mogli korzystać z bardziej zaawansowanych narzędzi, wspomagających zarządzanie czasem i osiąganie celów.

Praca może stanowić inspirację dla innych projektów związanych z wykorzystaniem sztucznej inteligencji w celu wspierania rozwoju osobistego i zawodowego.

\section{Skróty i symbole}

 \begin{enumerate}

    \item {\bf Giga } - Chad
    \item {\bf Giga } - Chad
     
 \end{enumerate}

\section{Faza analizy}
\subsection{Wstęp}

\indent Block elo

\clearpage

\subsection{Cele i założenia}

\subsection{Podstawa teoretyczna}

\subsection{Prompt engineering}

\subsubsection{Opis modeli językowych}

\subsubsection{Analiza przypdaków użycia modeli językowych}

\subsubsection{Projektowanie promptów}

\subsubsection{Prompt security}

\subsection{Platformy chmurowe}

\subsubsection{Opis platform}

\subsubsection{Wybór - Plaforma Microsoft Azure}

\subsection{Wymagania funkcjonalne}

\subsubsection{Opis funkcjonalności systemu}

\subsubsection{Diagram przypadków użycia}

\subsubsection{Scenariusze przypadków użycia}

\subsubsection{Sposób przechowywania danych}

\subsubsection{Diagramy}

\subsection{Wymagania niefunkcjonalne}

\subsection{Opis prototypów}




\section{Faza projektowania}

\subsection{Ogólny opis architektury}

\subsubsection{Mikroserwisy}

\subsection{DevOps}

\subsubsection{CI/CD}

\subsubsection{Kubernetes}

\subsubsection{Monitorowanie aplikacji}

\subsection{Baza danych}

\subsubsection{SQL}

\subsubsection{NoSQL}

\subsection{Model C4}

\subsection{Aplikacje front-endowe}

\subsubsection{Opis bibliotek i frameworków}

\subsubsection{Przykłady implementacji}

\subsection{Aplikacje back-endowe}

\subsubsection{Opis bibliotek/szkiele}

\subsubsection{Przykłady implementacji}

\subsection{Zastosowane praktyki bezpieczeństwa}

\include{7_podsumowanie}

\include{8_spis_ilustracji}

\section{Bibliografia}

\begin{enumerate}
\end{enumerate}


\end{document}