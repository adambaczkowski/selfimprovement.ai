\documentclass[12pt]{mwart}
\renewcommand{\baselinestretch}{1.5} 
\usepackage[utf8]{inputenc}
\usepackage{polski}
\usepackage{graphicx}
\usepackage{tocloft}
\usepackage{hyperref}

% ------------------------------------------------
%formatowanie kodu
\usepackage{listings}
\usepackage{xcolor}

%formatowanie obrazów
\usepackage{float}

% dodanie kolorów
\definecolor{codegreen}{rgb}{0,0.6,0}
\definecolor{codegray}{rgb}{0.5,0.5,0.5}
\definecolor{codepurple}{rgb}{0.58,0,0.82}
\definecolor{backcolour}{rgb}{0.95,0.95,0.92}

% definiowanie stylu kodu
\lstdefinestyle{mystyle}{
  backgroundcolor=\color{backcolour}, commentstyle=\color{codegreen},
  keywordstyle=\color{magenta},
  numberstyle=\tiny\color{codegray},
  stringstyle=\color{codepurple},
  basicstyle=\ttfamily\footnotesize,
  breakatwhitespace=false,         
  breaklines=true,                 
  captionpos=b,                    
  keepspaces=true,                 
  numbers=left,                    
  numbersep=5pt,                  
  showspaces=false,                
  showstringspaces=false,
  showtabs=false,                  
  tabsize=2
}

%"mystyle" code listing set
\lstset{style=mystyle}

% ------------------------------------------------

% Ustawienia akapitów
\setlength{\parindent}{1.5cm}

% Zmiana nazwy na "Spis treści"
\renewcommand*\contentsname{Spis treści}

\begin{document}

\begin{center}

    %University logo
    \includegraphics[width=1\textwidth]{Obrazy/PJATK_logo.png}
    \vspace{1cm}
    
     %Department
    {\normalsize Wydział Informatyki\par}
    \vspace{1cm}
    
    %Department
    {\normalsize Nazwa Katedry\par}
    {\bf{\normalsize Katedra Sieci Komputerowych i Systemów Rozproszonych\par}}
    \vspace{1cm}

    %Specialization
    {\normalsize Nazwa specjalizacji\par}
    {\bf{\normalsize Technologie sieci urządzeń mobilnych oraz chmury obliczeniowej\par}}
    \vspace{1cm}
    
    %Author's name
    {\bf\normalsize Grzegorz Święcicki,\\Adam Bączkowski,\\Mateusz Młodochowski\par}
    {\normalsize s20978, s20132, s29707\par}
    \vspace{0.8cm}
    
    %Thesis title
    {\bf\uppercase{\normalsize wykorzystanie sztucznej inteligencji w aplikacji internetowej do samodoskonalenia \par}}
    \vspace{1cm}
    
    %Thesis type
    {\raggedleft{\normalsize Praca magisterska \\ napisana pod opieką:\par}}
    {\raggedleft{\normalsize dr Pavel Savov\par}}
    {\raggedleft{\normalsize mgr. inż. Mokkas Mokkas\par}}
    \vspace{1cm}
    
    %Date
    {\normalsize Warszawa Lipiec 2024}

\end{center}
\clearpage

\begin{center}
    {\bf\Large{Streszczenie}}\\
\end{center}
Praca opisuje zaprojektowaną i wykonaną przez jej autorów implementację
aplikacji internetowej wspomaganej przez sztuczną inteligencję. Aplikacja dotyczy obszaru samodoskonalenia się i poprawiania swoich umiejętności w różnych dziedzinach życia. Aplikacja będzie stworzona w architekturze mikro serwisów oraz zbudowanie na podstawie najbardziej aktualnych wzorców projektowych.
Szczególny nacisk położono na integrację ze sztuczną inteligencją, która pomaga użytkownikowi aplikacji zaplanować plan rozwoju w zależności do jego preferencji.
Zespół składa się z programistów tworzących aplikację oraz inżyniera DevOps, automatyzującego procesy oraz planującego architekturę całego systemu. Jako programiści chcemy wykorzystać potencjał modeli językowych (LLM) w planowaniu codziennych zadań oraz obowiązków.
\\
{\bf Słowa kluczowe:} artificial intelligence, microservices, cloud, DevOps, LLM, fine tuning.
\clearpage

\begin{center}
    {\bf\Large{Abstrakt}}\\
\end{center}
The thesis describes the designed and implemented web application supported by artificial intelligence, created by its authors. The application focuses on the self-improvement and enhancement of skills in various life domains. It will be developed using a microservices architecture and built based on the most current design patterns. Special emphasis is placed on integration with artificial intelligence, which assists the application user in planning a development strategy according to their preferences.

The team consists of developers creating the application and a DevOps engineer automating processes and planning the architecture of the entire system. As developers, we aim to leverage the potential of language models (LLM) in planning daily tasks and responsibilities.
\\
{\bf Słowa kluczowe:} sztuczna inteligencja, mikroserwisy, chmura, DevOps, LLM, dostrajanie.

\tableofcontents

\section{Przedmowa}

\indent Przedstawiamy niniejszą pracę, która dokumentuje projektowanie i realizację aplikacji internetowej z wykorzystaniem wsparcia sztucznej inteligencji. Głównym celem tego projektu było opracowanie narzędzia umożliwiającego użytkownikom rozwój i doskonalenie w różnych dziedzinach życia. Zdecydowano się na wykorzystanie architektury mikroserwisów oraz najnowszych wzorców projektowych w celu zapewnienia efektywności, skalowalności i łatwości zarządzania systemem.

Szczególny nacisk położono na integrację sztucznej inteligencji, która pełni kluczową rolę w personalizacji planów rozwoju użytkowników. Aplikacja opiera się na zaawansowanych algorytmach, analizujących cele i preferencje użytkowników w celu proponowania spersonalizowanych strategii rozwoju.

Celem było nie tylko stworzenie funkcjonalnej aplikacji, ale również zapewnienie jej niezawodności, bezpieczeństwa i łatwości obsługi.

W trakcie realizacji projektu zdecydowano się na wykorzystanie potencjału modeli językowych (LLM) w celu usprawnienia planowania codziennych zadań i obowiązków. Dzięki temu użytkownicy będą mogli korzystać z bardziej zaawansowanych narzędzi, wspomagających zarządzanie czasem i osiąganie celów.

Praca może stanowić inspirację dla innych projektów związanych z wykorzystaniem sztucznej inteligencji w celu wspierania rozwoju osobistego i zawodowego.

\section{Skróty i symbole}

 \begin{enumerate}

    \item {\bf Giga } - Chad
    \item {\bf Giga } - Chad
     
 \end{enumerate}

\section{Faza analizy}
\subsection{Wstęp}

\indent Block elo

\clearpage

\subsection{Cele i założenia}

\subsection{Podstawa teoretyczna}

\subsection{Prompt engineering}

\subsubsection{Opis modeli językowych}

\subsubsection{Analiza przypdaków użycia modeli językowych}

\subsubsection{Projektowanie promptów}

\subsubsection{Prompt security}

\subsection{Platformy chmurowe}

\subsubsection{Opis platform}

\subsubsection{Wybór - Plaforma Microsoft Azure}

\subsection{Wymagania funkcjonalne}

\subsubsection{Opis funkcjonalności systemu}

\subsubsection{Diagram przypadków użycia}

\subsubsection{Scenariusze przypadków użycia}

\subsubsection{Sposób przechowywania danych}

\subsubsection{Diagramy}

\subsection{Wymagania niefunkcjonalne}

\subsection{Opis prototypów}




\section{Faza projektowania}

\subsection{Ogólny opis architektury}

\subsubsection{Mikroserwisy}

\subsection{DevOps}

\subsubsection{CI/CD}

\subsubsection{Kubernetes}

\subsubsection{Monitorowanie aplikacji}

\subsection{Baza danych}

\subsubsection{SQL}

\subsubsection{NoSQL}

\subsection{Model C4}

\subsection{Aplikacje front-endowe}

\subsubsection{Opis bibliotek i frameworków}

\subsubsection{Przykłady implementacji}

\subsection{Aplikacje back-endowe}

\subsubsection{Opis bibliotek/szkiele}

\subsubsection{Przykłady implementacji}

\subsection{Zastosowane praktyki bezpieczeństwa}

\section{Napotkane problemy}

\section{Rozwój}

\subsection{Plany na przyszłość}

\section{Podsumowanie}
\subsection{Stopień realizacji założeń}
W ramach niniejszej pracy magisterskiej zrealizowano wszystkie założenia, które zostały przedstawione na etapie planowania i projektowania. Głównym celem projektu było zaprojektowanie i implementacja aplikacji webowej wykorzystującej modele językowe dużej skali (LLM) do generowania planów działania wspierających użytkowników w osiąganiu zdefiniowanych celów.

Założenia projektowe obejmowały kilka kluczowych obszarów. Przede wszystkim, architektura aplikacji została zaplanowana w sposób umożliwiający wysoką skalowalność i efektywność, co zostało osiągnięte dzięki implementacji mikroserwisowej w chmurze Azure, wykorzystując technologie takie jak Azure Kubernetes Service. To pozwoliło na obsługę dużej liczby użytkowników jednocześnie bez utraty wydajności, spełniając założenia dotyczące skalowalności.

Interfejs użytkownika został zaprojektowany z myślą o prostocie, intuicyjności i przyjazności dla użytkowników o różnym stopniu zaawansowania technicznego. Wykorzystanie frameworka React umożliwiło stworzenie interaktywnego środowiska, które wspiera użytkowników w łatwym korzystaniu z funkcji aplikacji.

Bezpieczeństwo danych było priorytetem podczas całego procesu projektowania i implementacji. Wszystkie informacje użytkowników, w tym dane osobowe oraz szczegóły dotyczące celów i zadań, są przechowywane \linebreak i przetwarzane zgodnie z najnowszymi standardami bezpieczeństwa. Zastosowanie mechanizmów szyfrowania danych i zarządzania dostępem zapewnia pełną ochronę prywatności użytkowników.

Integracja modeli LLM z aplikacją została zrealizowana w sposób umożliwiający dynamiczne generowanie planów działania. Architektura aplikacji jest elastyczna i pozwala na łatwe dodawanie nowych modeli oraz aktualizację istniejących, co umożliwia aplikacji dostosowanie się do zmieniających się potrzeb i nowych technologii w dziedzinie sztucznej inteligencji.

Cała infrastruktura aplikacji, w tym mikroserwisy, bazy danych oraz usługi AI, została wdrożona w chmurze Azure, co zapewnia wysoką dostępność, niezawodność oraz łatwość zarządzania. Wybór technologii chmurowych był kluczowy dla realizacji założeń dotyczących infrastruktury, zapewniając jednocześnie możliwość szybkiego skalowania i adaptacji aplikacji do rosnących potrzeb użytkowników. Niestety ze względu na duże koszta nie możliwe było utrzymanie tej architektury dla tego została ona postawiona jedynie\linebreak w celach naukowych na kilka dni. Każdy z celów szczegółowych oraz założeń technicznych został skrupulatnie spełniony, co pozwala na wyciągnięcie kilku istotnych wniosków.
\subsection{Wnioski Końcowe}

Niniejsza praca magisterska miała na celu zaprojektowanie i implementację aplikacji webowej, która wykorzystuje modele językowe dużej skali (LLM) do generowania planów działania, wspierających użytkowników w osiąganiu ich celów, takich jak przebiegnięcie maratonu czy nauka programowania. Aplikacja ta integruje nowoczesne technologie frontendowe \linebreak i backendowe oraz architekturę mikroserwisową opartą na chmurze Azure, co zapewnia jej skalowalność, niezawodność oraz efektywność.

W ramach pracy przeprowadzono szczegółową analizę dostępnych modeli LLM, takich jak Chat GPT, LLama oraz Zephyre, i porównano ich efektywność w kontekście generowania planów. Dzięki temu możliwe było wybranie najbardziej odpowiednich modeli do integracji z aplikacją.

Interfejs użytkownika został zbudowany przy użyciu biblioteki React, co zapewniło modularność, elastyczność oraz łatwość utrzymania kodu. Użycie narzędzi takich jak Formik do zarządzania formularzami oraz React Query do zarządzania zapytaniami HTTP i synchronizacją danych z serwerem znacznie uprościło proces tworzenia aplikacji.

Backend aplikacji, oparty na języku C\# i frameworku ASP.NET Core, został zaprojektowany jako zestaw mikroserwisów, co umożliwia łatwe zarządzanie poszczególnymi funkcjonalnościami oraz ich skalowanie w miarę potrzeb. Wdrożenie mikroserwisów w środowisku chmurowym Azure przyczyniło się do zapewnienia wysokiej dostępności i niezawodności systemu.

Podczas tworzenia aplikacji napotkano pewne problemy. Zrezygnowano z Kubernetesa oraz Azure Kubernetes Service, ponieważ koszty okazały się nieproporcjonalne do obecnego stanu i liczby użytkowników aplikacji.\linebreak W rezultacie, aplikacja jest obecnie oparta wyłącznie na kontenerach Dockerowych oraz innych modułach platformy Azure. Ponadto, lokalne modele LLM często działały wyłącznie na procesorze zamiast na karcie graficznej z powodu nieodpowiedniej konfiguracji, co powodowało duże opóźnienia\linebreak w otrzymywaniu odpowiedzi na zapytania. 

Dużym problemem okazały się również koszty utrzymania infrastruktury na środowisku deweloperskim. Naszym budżetem na projekt było około 200 złotych, jednak analizując prognozę kosztów w Azure, nasz budżet szybko został przekroczony. Największe koszty wygenerowały instancje kontenerów Dockerowych działające w chmurze.

Szczególną uwagę poświęcono również aspektom autoryzacji i uwierzytelniania użytkowników, które są kluczowe dla bezpieczeństwa aplikacji. Implementacja mechanizmów logowania, rejestracji oraz zarządzania sesjami użytkowników zapewniła bezpieczne korzystanie z aplikacji.

W trakcie realizacji projektu napotkano wiele wyzwań, takich jak integracja różnych technologii, zapewnienie wydajności i skalowalności systemu oraz optymalizacja interfejsu użytkownika pod kątem użyteczności\linebreak i responsywności. Dzięki zastosowaniu nowoczesnych narzędzi i najlepszych praktyk inżynierii oprogramowania, udało się stworzyć aplikację, która nie tylko spełnia założone cele, ale także stanowi solidną podstawę do dalszego rozwoju.

Podsumowując, praca ta dostarczyła wartościowych informacji na temat wykorzystania modeli LLM w praktycznych zastosowaniach oraz przyczyniła się do rozwoju wiedzy na temat projektowania i implementacji nowoczesnych aplikacji webowych. Wyniki pracy mogą być podstawą do dalszych badań oraz inspiracją do tworzenia innowacyjnych rozwiązań wspierających użytkowników w osiąganiu ich celów.

\include{9_spis_ilustracji}

\section{Bibliografia}

\begin{enumerate}

\item {\textit {Courtemanche, Meredith; Mell, Emily; Gills, Alexander S. "What Is DevOps? The Ultimate Guide". TechTarget. [Dostęp: 2023-01-22]}}
\item {\textit {Ray, Partha Pratim. “An Introduction to Dew Computing: Definition, Concept and Implications.” IEEE Access 6 (2018): 723-737.}}

\item {\textit {"Better Language Models and Their Implications". OpenAI. 2019-02-14. Archived from the original on 2020-12-19.[Dostęp: 2019-08-25]}}
\item {\textit {Russell, Stuart J.; Norvig, Peter. (2021). Artificial Intelligence: A Modern Approach (4th ed.). Hoboken: Pearson. ISBN 978-0134610993. LCCN 20190474}}
\item {\textit {Reddy, Martin (2011). API Design for C++. Elsevier Science. p. 1. ISBN 9780123850041. [Dostęp: 2023-03-21]}}
\item {\textit {Don, Norman; Jakob, Nielsen. "The Definition of User Experience (UX)". Nielsen Norman Group. [Dostęp: 2 March 2021.]}}
\item {\textit {HTML: HyperText Markup Language | MDN \\ https://developer.mozilla.org/en-US/docs/Web/HTML}}
\item {\textit {"CSS developer guide". MDN Web Docs. \\ https://developer.mozilla.org/en-US/docs/Learn/CSS}}
\item {\textit {Sztuczna inteligencja w pracy informatyka. Marcin Szeliga \\ https://itprofessional.pl/2024/04/05/sztuczna-inteligencja-w-pracy-informatyka/}}
\item {\textit {Sztuczna inteligencja albo nas zbawi, albo zabije. Mam tyle samo nadziei, co obaw. Michał Mielnik. \\ https://www.chip.pl/2023/03/sztuczna-inteligencja-szanse-i-zagrozenia}}
\item {\textit {What is ChatGPT?. https://help.openai.com/en/articles/6783457-what-is-chatgpt}}
\item {\textit {Gabe Zichermann, Christopher Cunningham, "Gamification by Design Implementing Game Mechanics in Web and Mobile Apps" O'Reilly Media 2011-08}}
\item {\textit {Introducing LLaMA: A foundational, 65-billion-parameter large language model \\ https://ai.meta.com/blog/large-language-model-llama-meta-ai/}}
\item {\textit {Stable LM Zephyr 3B: A Powerful and Efficient Language Model for Edge Devices. Mirza Causevic. \\ https://medium.com/@mizcausevic/stable-lm-zephyr-3b-a-powerful-and-efficient-\\language-model-for-edge-devices-14bb2de04c91}}
\item {\textit {Czym jest prompt engineering i co pomoże Ci w pisaniu dobrych promptów. Paweł Marszalec. https://geek.justjoin.it/czym-jest-prompt-engineering\\-i-co-pomoze-ci-w-pisaniu-dobrych-promptow/}}
\item {\textit {ArtPrompt: ASCII Art-based Jailbreak Attacks against Aligned LLMs.Jiang, Fengqing and Xu, Zhangchen and Niu, Luyao and Xiang, Zhen and Ramasubramanian, Bhaskar and Li, Bo and Poovendran, Radha. \\ https://arxiv.org/pdf/2402.11753 }}
\item {\textit {Baseline Defenses for Adversarial Attacks Against Aligned Language Models. Jain, Neel and Schwarzschild, Avi and Wen, Yuxin and Somepalli, Gowthami and Kirchenbauer, John and Chiang, Ping-yeh and Goldblum, Micah and Saha, Aniruddha and Geiping, Jonas and Goldstein, Tom. https://arxiv.org/pdf/2309.00614}}
\item {\textit {Quinn, Joanne (2020). Dive into deep learning: tools for engagement. Thousand Oaks, California. p. 551. ISBN 978-1-5443-6137-6}}
\item {\textit {Why ChatGPT and Bing Chat are so good at making things up. Benj Edwards. \\ https://arstechnica.com/information-technology/2023/04/why-ai-chatbots-are-\\the-ultimate-bs-machines-and-how-people-hope-to-fix-them/}}
\item {\textit {The 15 Biggest Risks Of Artificial Intelligence. Bernard Marr. \\ https://www.forbes.com/sites/bernardmarr/2023/06/02/the-15-biggest-risks\\-of-artificial-intelligence/}}
\item {\textit {P. P. Ray, "An Introduction to Dew Computing: Definition, Concept and Implications," in IEEE Access, vol. 6, pp. 723-737, 2018, doi: 10.1109/ACCESS.2017.2775042}}
\item {\textit {Co to jest chmura obliczeniowa? \\ https://www.oracle.com/pl/cloud/what-is-cloud-computing/}}
\item {\textit {Mikrousługi na platformie Azure. \\ https://azure.microsoft.com/pl-pl/solutions/\\microservice-applications}}
\item {\textit {Paweł Kamiński. “React. Wstęp do programowania" Wydawnictwo - Helion, 2021}}
\item {\textit {Alex Banks, Eve Porcello. "Learning React: Modern Patterns for Developing React Apps". O'Reilly 2020-06-12}}
\item {\textit {Mikrousługi .NET: architektura konteneryzowanych aplikacji .NET. \\ https://learn.microsoft.com/pl-pl/dotnet/architecture/microservices/}}
\item {\textit{Roger Villela. "Exploring the .NET Core 3.0 Runtime". Apress 2019-09-06}}
\item {\textit {Prabath Siriwardena. "Advanced API Security. OAuth 2.0 and Beyond". Apress 2019-12-16}}
\item {\textit {Saineshwar Bageri. "RabbitMQ With C\#". BPB PUBN 2018-11-08}}

\end{enumerate}


\end{document}