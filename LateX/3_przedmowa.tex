\section{Przedmowa}

\indent Przedstawiamy niniejszą pracę, która dokumentuje projektowanie i realizację aplikacji internetowej z wykorzystaniem wsparcia sztucznej inteligencji. Głównym celem tego projektu było opracowanie narzędzia umożliwiającego użytkownikom rozwój i doskonalenie w różnych dziedzinach życia. Zdecydowano się na wykorzystanie architektury mikroserwisów oraz najnowszych wzorców projektowych w celu zapewnienia efektywności, skalowalności i łatwości zarządzania systemem.

Szczególny nacisk położono na integrację sztucznej inteligencji, która pełni kluczową rolę w personalizacji planów rozwoju użytkowników. Aplikacja opiera się na zaawansowanych algorytmach, analizujących cele i preferencje użytkowników w celu proponowania spersonalizowanych strategii rozwoju.

Celem było nie tylko stworzenie funkcjonalnej aplikacji, ale również zapewnienie jej niezawodności, bezpieczeństwa i łatwości obsługi.

W trakcie realizacji projektu zdecydowano się na wykorzystanie potencjału modeli językowych (LLM) w celu usprawnienia planowania codziennych zadań i obowiązków. Dzięki temu użytkownicy będą mogli korzystać z bardziej zaawansowanych narzędzi, wspomagających zarządzanie czasem i osiąganie celów.

Praca może stanowić inspirację dla innych projektów związanych z wykorzystaniem sztucznej inteligencji w celu wspierania rozwoju osobistego i zawodowego.