\section{Przedmowa}

\indent Motywacją do sworzenia niniejszej pracy, jest szybki rozwój technologii sztucznej inteligencjii oraz chatbotów. Głównym celem  projektu było opracowanie narzędzia umożliwiającego użytkownikom rozwój i doskonalenie się w różnych dziedzinach życia. Zdecydowano się na wykorzystanie architektury mikroserwisów oraz najnowszych wzorców projektowych w celu zapewnienia efektywności, skalowalności i łatwości zarządzania całą architekturą.

Szczególny nacisk położono na integrację sztucznej inteligencji, która pełni kluczową rolę w personalizacji planów rozwoju użytkowników. Aplikacja opiera się na zaawansowanych modelach językowych, analizujących cele\linebreak i preferencje użytkowników w celu proponowania spersonalizowanych strategii rozwoju.

Celem było stworzenie funkcjonalnej aplikacji oraz również zapewnienie jej niezawodności, bezpieczeństwa i łatwości obsługi.

W trakcie realizacji projektu zdecydowano się na wykorzystanie potencjału modeli językowych w celu usprawnienia planowania codziennych zadań i obowiązków. Dzięki temu użytkownicy będą mogli korzystać z bardziej zaawansowanych narzędzi, wspomagających zarządzanie czasem i osiąganie celów.

Praca może stanowić inspirację dla innych projektów związanych\linebreak z wykorzystaniem sztucznej inteligencji w celu wspierania rozwoju osobistego i zawodowego.