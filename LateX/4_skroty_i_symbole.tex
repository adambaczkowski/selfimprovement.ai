\section{Skróty i symbole}

 \begin{enumerate}
    \item {\bf DevOps} - to praktyka łącząca działania zespołów developerskich i operacyjnych w celu automatyzacji i przyspieszenia procesów wytwarzania oprogramowania oraz wdrożenia aplikacji. Celem jest poprawa efektywności, niezawodności i elastyczności dostarczania usług poprzez integrację ciągłą, dostarczanie ciągłe i automatyzację procesów[1].
    
    \item {\bf Cloud} - termin odnoszący się do dostarczania usług informatycznych, takich jak moc obliczeniowa, współdzielenie pamięci masowej lub serwisów np. baza danych jako usługa. Zamiast lokalnie przechowywać dane czy uruchamiać aplikacje na komputerach osobistych, korzysta się z zasobów udostępnianych przez dostawców usług chmurowych, co umożliwia elastyczność, skalowalność oraz dostępność usług na żądanie w różnych regionach świata[2].
    
    \item {\bf LLM} - Large Language Models - Duże Modele Językowe, model sztucznej inteligencji umożliwiający wszechstronne generowanie tekstu oraz realizację zadań związanych z przetwarzaniem języka naturalnego. Modele LLM są szkolone w ramach samo nadzorowanego lub słabo nadzorowanego uczenia maszynowego z wykorzystaniem dużych ilości danych tekstowych. Proces ten jest bardzo intensywny obliczeniowo i wymaga dużych zasobów obliczeniowych[3].
    
    \item {\bf AI} - Artificial Intelligence - Czyli Sztuczna Inteligencja. Pojęcie obejmujące dziedzinę informatyki, która rozwija systemy komputerowe zdolne do wykonywania zadań, zazwyczaj wymagających ludzkiej inteligencji. Przykładowymi zadaniami mogą być analiza danych, podejmowanie decyzji czy komunikacja z inną osobą[4].
    
    \item {\bf API} - Application Programming Interface - interfejs programowania aplikacji to zestaw reguł i definicji, które umożliwiają aplikacjom komunikację między sobą[5].
    
    \item {\bf UX} - User Experience - proces projektowania produktu, który ma na celu zwiększenie komfortu i satysfakcji użytkownika[6].
    
    \item {\bf UI} - User Interface - interfejs użytkownika, czyli sposób w jaki użytkownik komunikuje się z produktem.
   
    \item {\bf HTML (HyperText Markup Language)} - Język znaczników używany do tworzenia struktur i treści stron internetowych. Składa się z zestawu znaczników, które definiują różne elementy na stronie, takie jak nagłówki, paragrafy, listy, obrazy, linki itp. HTML określa strukturę dokumentu, pozwalając przeglądarką internetowym na interpretację i wyświetlanie zawartości w odpowiedni sposób. Jest to fundamentalny język używany wraz z CSS i JavaScriptem do tworzenia interaktywnych i atrakcyjnych stron internetowych[7].
    
    \item {\bf CSS (Cascading Style Sheets)} - jest językiem stylów używanym do definiowania wyglądu i prezentacji dokumentów HTML. CSS określa sposób, w jaki elementy HTML są wyświetlane na stronie internetowej, regulując kolor, rozmiar, odstępy, animacje i wiele innych właściwości. Jest to niezbędny element tworzenia interfejsów użytkownika w aplikacjach webowych, pozwalający na nadanie strukturze HTML odpowiedniego wyglądu i formatowania. CSS jest wspierany przez wszystkie przeglądarki internetowe i stanowi podstawę stylizacji stron internetowych na całym świecie[8].

    \item {\bf Pipeline} - Pipeline to zautomatyzowany proces składający się z serii kroków (zadań), służący do ciągłej integracji, testowania, wdrażania i dostarczania oprogramowania, umożliwiając sprawne i spójne wypuszczanie aktualizacji produktu. Jako źródło danych wykorzystuje repozytorium Git np. Github.

    \item {\bf GitOps} - Najpopularniejszym systemem kontroli wersji jest narzędzie Git, podejście GitOps zostało nazwane więc na jego cześć. Praktyka wywodząca się z kultury DevOps w myśl, ze zmianami konfiguracji można zarządzać za pomocą praktyk przeglądu kodu i można je wycofać za pomocą kontroli wersji. Zasadniczo wszystkie zmiany są śledzone i zapisywane w repozytorium kodu, co następnie daje możliwość tworzenia\linebreak i modyfikowania infrastruktury.

    \item {\bf Infrastructure as Code (IaC)} - to proces zarządzania i udostępniania zasobów komputerowych za pomocą plików konfiguracyjnych nadających się do odczytu maszynowego. Jest to przeciwieństwo do starszej praktyki fizycznej konfiguracji sprzętu lub interaktywnych narzędzi konfiguracyjnych. Infrastruktura IT zarządzana przez ten proces obejmuje zarówno sprzęt fizyczny np. serwery "bare-metal", jak i maszyny wirtualne oraz powiązane zasoby konfiguracyjne. Definicje mogą znajdować się w systemie kontroli wersji. Kod w plikach definicji może wykorzystywać skrypty lub definicje deklaratywne, ale IaC częściej stosuje podejścia deklaratywne. Jednym z najpopularniejszych narzędzi jest Terraform.
     
 \end{enumerate}