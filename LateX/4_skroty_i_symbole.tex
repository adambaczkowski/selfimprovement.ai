\section{Skróty i symbole}

 \begin{enumerate}
    \item {\bf DevOps} - to praktyka łącząca działania zespołów developerskich i operacyjnych w celu automatyzacji i przyspieszenia procesów wytwarzania oprogramowania oraz wdrożenia aplikacji. Celem jest poprawa efektywności, niezawodności i elastyczności dostarczania usług poprzez integrację ciągłą, dostarczanie ciągłe i automatyzację procesów.
    
    \item {\bf Cloud} -  termin odnoszący się do dostarczania usług informatycznych, takich jak przechowywanie danych, obliczenia czy dostęp do aplikacji, poprzez internet. Zamiast lokalnie przechowywać dane czy uruchamiać aplikacje na własnych komputerach, korzysta się z zasobów udostępnianych przez dostawców usług chmurowych, co umożliwia elastyczność, skalowalność oraz dostępność usług na żądanie.
    
    \item {\bf LLM} - Large Language Models - Duże Modele Językowe
    
    \item {\bf AI} - Artificial Intelligence - Czyli Sztuczna Inteligencja. Obejmuje dziedzinę informatyki, która rozwija systemy komputerowe zdolne do wykonywania zadań, które zazwyczaj wymagają ludzkiej inteligencji, takie jak analiza danych, podejmowanie decyzji czy komunikacja.
    
    \item {\bf API} - Application Programming Interface - interfejs programowania aplikacji to zestaw reguł i definicji, które umożliwiają aplikacjom komunikację między sobą.
    
    \item {\bf UX} - User Experience - proces projektowania produktu, który ma na celu zwiększenie komfortu i satysfakcji użytkownika
    
    \item {\bf UI} - User Interface - interfejs użytkownika, czyli sposób w jaki użytkownik komunikuje się z produktem
   
    \item {\bf HTML (HyperText Markup Language)} - język znaczników używany do tworzenia struktur i treści stron internetowych. Składa się z zestawu znaczników, które definiują różne elementy na stronie, takie jak nagłówki, paragrafy, listy, obrazy, linki itp. HTML określa strukturę dokumentu, pozwalając przeglądarkom internetowym na interpretację i wyświetlanie zawartości w odpowiedni sposób. Jest to fundamentalny język używany wraz z CSS i JavaScriptem do tworzenia interaktywnych i atrakcyjnych stron internetowych.
    
    \item {\bf CSS (Cascading Style Sheets)} - jest językiem stylów używanym do definiowania wyglądu i prezentacji dokumentów HTML. CSS określa sposób, w jaki elementy HTML są wyświetlane na stronie internetowej, regulując kolor, rozmiar, odstępy, animacje i wiele innych właściwości. Jest to niezbędny element tworzenia interfejsów użytkownika w aplikacjach webowych, pozwalający na nadanie strukturze HTML odpowiedniego wyglądu i formatowania. CSS jest wspierany przez wszystkie przeglądarki internetowe i stanowi podstawę stylizacji stron internetowych na całym świecie.

    \item {\bf Pipeline} - Pipeline to zautomatyzowany proces składający się z serii kroków (zadań), który służy do ciągłej integracji, testowania, wdrażania i dostarczania oprogramowania, umożliwiając sprawne i spójne wypuszczanie aktualizacji produktu.
     
 \end{enumerate}